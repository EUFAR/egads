%% $Date: 2012-07-06 17:42:54#$
%% $Revision: 176 $
\index{diameter\_median\_volume\_dmt}
\algdesc{Median Volume Diameter}
{ %%%%%% Algorithm name %%%%%%
diameter\_median\_volume\_dmt
}
{ %%%%%% Algorithm summary %%%%%%
Calculates the median volume diameter given a size distribution. The median volume diameter
is the size of droplet below which 50\% of the total water volume resides. 
}
{ %%%%%% Category %%%%%%
Microphysics
}
{ %%%%%% Inputs %%%%%%
$c_i$ & Array[time, bins] & Number concentration of hydrometeors in size category $i$ [cm$^{-3}$] \\
$d_i$ & Vector[bins] & Average diameter of size category $i$ [$\mu$m] \\
$s_i$ & Array[time, bins],Optional & Shape factor of the hydrometeor of size category $i$ to account for asphericity \\
$\rho_i$ & Vector[bins],Optional & Density of hydrometeor in size category $i$ [g cm$^{-3}$]. Default is $\rho_w$ = 1.0 g cm$^-3$
}
{ %%%%%% Outputs %%%%%%
$D_{mvd}$ & Vector[time] & Median volume diameter [$\mu$m]
}
{ %%%%%% Formula %%%%%%
Step 1: Compute liquid water content
%
\begin{displaymath}
 W = \frac{\pi}{6} \sum \limits_{i=1}^m{c_i d_i^3 \rho_i s_i}
\end{displaymath}

Step 2: Beginning at the first size channel, calculate the accumulated mass $S_n = w_1 + w_2 + ... w_n$
where $w_1$ is the mass of water in channel 1, and $w_n$ is the channel where the accumulated mass is
greater than or equal to 0.5$W$, i.e. greater than or equal to 50\% of the total LWC.

Step 3: Compute the median volume diameter, $D_{mvd}$ by interpolating linearly between the channels
that bracket where the accumulated mass exceeded the total LWC:
%
\begin{displaymath}
 D_{mvd} = d_{n-1} + (0.5 - S_{n-1}/S_n) (d_n - d_{n-1})
\end{displaymath}
 
}
{ %%%%%% Author %%%%%%

}
{ %%%%%% References %%%%%% 
    ``Data Analysis User's Guide Chapter I: Single Particle Light Scattering,`` Droplet Measurement Technologies, 33. \cite{DMT1}
}


