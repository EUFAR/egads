%% $Date: 2012-07-06 17:42:54#$
%% $Revision: 152 $
\index{temp\_blackbody}
\algdesc{Blackbody Temperature}
{ %%%%%% Algorithm name %%%%%%
temp\_blackbody
}
{ %%%%%% Algorithm summary %%%%%%
Calculates the blackbody temperature for a given radiance at a specific wavelength.
}
{ %%%%%% Category %%%%%%
Radiation
}
{ %%%%%% Inputs %%%%%%
$rad$ & Vector & Blackbody radiance [W m-2 sr-1 nm-1] \\
$\lambda$ & Coeff & Wavelength [nm] \\
}
{ %%%%%% Outputs %%%%%%
$T$ & Vector & Temperature [K] \\
}
{ %%%%%% Formula %%%%%%
After converting $\lambda$ to m and $rad$ to W m-3 sr-1, the blackbody temperature is calculated by:
%
\begin{displaymath}
T = \frac{h c}{k_B \lambda \ln(\frac{2 h c^2}{\lambda^5 rad} +1)}
\end{displaymath}

where $c$ is the speed of light in m/s, $h$ is the Planck constant in J s and $k_B$ is the Boltzmann
constant in J/K.

}
{ %%%%%% Author %%%%%%
Andre Ehrlich, Leipzig Institute for Meteorology (a.ehrlich@uni-leipzig.de)

}
{ %%%%%% References %%%%%% 

}


